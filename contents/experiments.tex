\section{Experiments}

\subsection{Random forest}
% https://www.tensorflow.org/decision_forests
TensorFlow Decision Forests (TF-DF) is the library used to train and evaluate the random forest model.

% (https://www.stat.berkeley.edu/~breiman/randomforest2001.pdf)
A Random Forest is a collection of deep CART decision trees trained independently and without pruning.
Each tree is trained on a random subset of the original training dataset (sampled with replacement).
The algorithm is unique in that it is robust to overfitting, even in extreme cases e.g. when there are more features than training examples.
It is probably the most well-known of the Decision Forest training algorithms.

\subsubsection{Features}
We need to adjust the dataset because the Random Forest model does not accept time series or multivariate data as input.

\begin{figure}[!htbp]
  \begin{subfigure}{.5\textwidth}
    \centering
    \[
      \begin{blockarray}{ccccc}
        & b_0 & b_1 & \dots & b_{15} \\
        \begin{block}{c|cccc|}
          t_0 & 0.2 & 1.6 & \dots & 1.7  \\
          t_1 & 1.3 & 1.8 & \dots & 1.8 \\
          \vdots & \vdots & \vdots &  & \vdots   \\
          t_{53} & 0.6 & 0.4 & \dots & 1.3 \\
        \end{block}
      \end{blockarray}
    \]
    \caption{One observation in a 2-dim array}
    \label{fig:figtrans1}
  \end{subfigure}%
  \begin{subfigure}{.5\textwidth}
    \centering
    \[
      \begin{blockarray}{cc}
      \begin{block}{c|c|}
        t_0::b_0 & 0.2 \\
        t_0::b_1 & 1.6 \\
        \vdots & \vdots \\
        t_{53}::b_{15} & 1.3 \\
      \end{block}
      \end{blockarray}
    \]
    \caption{Forged features for Random forest}
    \label{fig:figtrans2}
  \end{subfigure}
  \caption{Transformation of one observation from a 2-dim array to a 1-dim array}
  \label{fig:figtrans}
\end{figure}

As shown in Figure \ref{fig:figtrans1}, each sample is initially represented by a 2-dimensional array of size (54, 16), where 16 bands capture the pixel's characteristics and 54 time steps reflect its progression. 
However, through the transformation process, the representation is transformed into a more 1-dimensional array of size (864, 1) as depicted in Figure \ref{fig:figtrans2}. 

The model's training utilizes the new representation of the data, composed of 864 derived features, to generate 300 trees.
The performance of each tree is then summarized in Table \ref{tab:rfresults}, which displays the number of nodes, the number of features utilized, and the overall accuracy

\begin{table}[!htbp]
  \centering
    \begin{tabular}{lrrr}
    Model                       & Nodes   & Used features & Overall Accuracy             \\[0.2cm] 
    \hline \\[-0.2cm]
    Imputed missing values      & 523,636  & 753          & $91.03 \pm 0.42$\\
    Not imputed missing values  & 519,932  & 718          & $89.72 \pm 0.84$
    \end{tabular}
  \caption{Random forest results}
  \label{tab:rfresults}
\end{table}

\subsection{Temp CNN}
- model overview

\subsection{AJ-RNN}
- model overview\\
- GRU vs LSTM\\
- learnig rate\\
- batch size\\

\subsubsection{Light AJ-RNN}
- model overview\\
- baseline

\subsection{L-TAE}
- model overview\\
- PixelSetEncoder\\
- DenseEncoder

\subsection{Results}

- table with accuracy