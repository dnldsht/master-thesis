\section{Random forest}

TensorFlow Decision Forests (TF-DF) \cite{TensorFlow:rf} is the library used to train and evaluate the random forest model.

A Random Forest \cite{breiman2001random} is a collection of deep CART decision trees trained independently and without pruning.
Each tree is trained on a random subset of the original training dataset (sampled with replacement).
The algorithm is unique in that it is robust to overfitting, even in extreme cases e.g. when there are more features than training examples.
It is probably the most well-known of the Decision Forest training algorithms.

\subsection{Data preparation}
We need to adjust the dataset because the Random Forest model does not accept time series or multivariate data as input.

\begin{figure}[H]
  \begin{subfigure}{.49\textwidth}
    \centering
    \[
      \begin{blockarray}{ccccc}
        & b_0 & b_1 & \dots & b_{15} \\
        \begin{block}{c|cccc|}
          t_0 & 0.2 & 1.6 & \dots & 1.7  \\
          t_1 & 1.3 & 1.8 & \dots & 1.8 \\
          \vdots & \vdots & \vdots &  & \vdots   \\
          t_{53} & 0.6 & 0.4 & \dots & 1.3 \\
        \end{block}
      \end{blockarray}
    \]
    \caption{One observation in a 2-dim array}
    \label{fig:figtrans1}
  \end{subfigure}
  \begin{subfigure}{.49\textwidth}
    \centering
    \[
      \begin{blockarray}{cc}
      \begin{block}{c|c|}
        t_0::b_0 & 0.2 \\
        t_0::b_1 & 1.6 \\
        \vdots & \vdots \\
        t_{53}::b_{15} & 1.3 \\
      \end{block}
      \end{blockarray}
    \]
    \caption{One observation in a 1-dim array}
    \label{fig:figtrans2}
  \end{subfigure}
  \caption{Transformation of one observation from a 2-dim array to a 1-dim array}
  \label{fig:figtrans}
\end{figure}

As shown in Figure \ref{fig:figtrans1}, each sample is initially represented by a 2-dimensional array of size (54, 16), where 16 bands capture the characteristics of the pixel and 54 time steps reflect its evolution. 
However, the transformation process transforms the representation into a 1-dimensional array of size (864, 1), as shown in Figure \ref{fig:figtrans2}. 

The training of the model uses the new representation of the data, composed of 864 inferred features, to generate 300 trees.
Table \ref{tab:rfresults} shows the results of the Random Forest model trained with and without pre-imputed missing values. 
The table includes the number of nodes, the number of features used, and the overall accuracy for each case.

\begin{table}[H]
  \centering
    \begin{tabular}{lrrr}
    Case                       & Nodes   & Used features & Overall Accuracy             \\[0.2cm] 
    \hline \\[-0.2cm]
    Pre imputation      & 523,636  & 753          & $91.03 \pm 0.42$\\
    No imputation       & 519,932  & 718          & $89.72 \pm 0.84$
    \end{tabular}
  \caption{Random forest results}
  \label{tab:rfresults}
\end{table}