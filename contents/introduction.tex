\chapter{Introduction}

% %- timeseries classification interesting topic
% % - abundant satellite images
% Time series classification is a challenging and interesting topic that has gained significant attention over the past few years.
% With the availability of abundant satellite images, time series classification has become increasingly important in several applications such as remote sensing, climate monitoring, and environmental studies.
% The classification of time series data is a crucial task that can help extract meaningful information and insights from the data.
% % TODO expand?

Recent advances in satellite imaging technology, such as the European Union's Sentinel program, have made it possible to acquire images with a high revisit rate.
As a result, Satellite Image Time Series (SITS) data have become increasingly abundant and can be used to describe how natural and semi-natural areas evolve over time.
For example, SITS data can be used in remote sensing applications for land cover classification and change detection, climate monitoring, and environmental studies.

% TODO rephrase
However, applying supervised learning techniques to SITS data requires a reference dataset.
This means that humans must manually label the data, which can be time-consuming and labor-intensive, but is necessary to extract valuable insights and information from the data.


%- different models/methods for classification
This thesis focuses on the development and evaluation of deep learning models for the classification of time series satellite images.
The proposed models aim to classify the images accurately and efficiently, while addressing some of the challenges associated with classifying time series data.

% talk about challenges
One of the challenges we faced during our experiments was dealing with missing values in the dataset. 
Since our dataset consists of time series data of satellite images, it was expected that there would be some missing values due to weather conditions or technical issues.
Missing values can affect the performance of deep learning models because they rely on a complete set of data to learn and make accurate predictions.
Since missing values can affect the performance of our models, we approached this challenge in two ways: first, by using imputation techniques to fill in the missing values to ensure that our models could still perform well with incomplete data; second, by conducting experiments without imputation to evaluate the impact of missing values on the performance of our models.

% TODO list models?
The models used in our study include Random Forests, Temporal Convolutional Neural Networks (TempCNNs), Recurrent Neural Networks (RNNs), Generative Adversarial Networks (GANs), and Transformers.

% Talk about chapters contents:
The thesis is organized as follows: Chapter 2 provides background information on time series and deep learning models.
Chapter 3 describes the dataset used in the experiments, including its characteristics and properties.
In Chapter 4, we present our proposed method to compare the performance of different deep learning models for time series classification tasks. 
In Chapter 5, we provide a detailed description of the models used in our study, including their architectures, hyperparameters, training procedures, experiments, and the results we obtained from these experiments.
Finally, in Chapter 6, we present our conclusions and suggest possible future work in the area of time series classification.