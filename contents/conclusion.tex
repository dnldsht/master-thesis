\chapter{Conclusions}

Experimenting with all the models presented in this work has been a challenging but rewarding experience.
It required a significant amount of time and effort to understand the underlying concepts of each model and adapt them to the specific problem at hand.
However, this process has greatly improved my skills in machine learning and data analysis, as well as my ability to critically evaluate and compare different models.
Overall, this work has provided me with valuable insights into the application of various deep learning models for time series classification tasks, and has expanded my knowledge in the field.

\section{Future work}
In the future, there are several avenues for further research that can build upon the work presented in this study.
Firstly, additional evaluation metrics could be included for a more comprehensive analysis of model performance.
For example, F1-Score and Mean IoU are widely used metrics in image classification and segmentation tasks and could provide more insights into the performance of the models.

Another area of research could be graph neural networks (GNNs), which have recently gained popularity in various domains, including computer vision and natural language processing. 
GNNs are particularly useful for data with graph structures, such as social networks, chemical compounds, and 3D objects. 
It would be interesting to explore the potential of GNNs for time series classification tasks and how they compare to traditional deep learning models.

Another potential direction for future work is the use of Im-BiLSTM, a novel architecture proposed by Chen et al. \cite{CHEN2022102762} for time series classification tasks.
Im-BiLSTM use bidirectional LSTM layers to enhance the learning of long-term dependencies and improve the interpretability of the model. 
It would be interesting to compare the performance of Im-BiLSTM with the models evaluated in this study.

Finally, transfer learning could be another area of future research for time series classification tasks.
Transfer learning has been successful in various computer vision tasks and involves leveraging pre-trained models to improve the performance of a new task with limited training data. 
It would be interesting to investigate the potential of transfer learning for time series classification and how it can be applied to the models evaluated in this study.