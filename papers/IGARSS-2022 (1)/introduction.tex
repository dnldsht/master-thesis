Nowadays, satellite imagery represents a fundamental source of information to monitor the dynamic of the Earth surface providing valuable knowledge to support decision makers in several application domains~\cite{TOTH201622}. Recent spatial programmes (i.e. the European Union's Sentinel program) provide images acquisition with high revisit time period. This produces Satellite Image Time Series (SITS) data that can be leveraged to describe how natural and semi-natural areas evolve over time~\cite{GOMEZ201655}. 

To this end, SITS data is regularly used as input to modern supervised machine learning approaches with the aim to provide up to date Land Cover Maps (LCM) over a specific region~\cite{GOMEZ201655}. SITS data, conversely to mono-date imagery, contains information about the evolution of the signal depicting the Earth surface allowing, for instance, to distinguish land covers that evolve differently over time (i.e. soybeans vs. corn crops due to the fact that they exhibit a different dynamic in their radiometric signal over time).

Unfortunately, supervised machine learning methods require large amount of reference (or Ground Truth (GT)) data to be calibrated, hence posing serious challenges to their use in situations characterized by a reduced amount of or unavailable reference data. For instance, when LCM should be updated from an year to a successive one, costs related to human-effort and resources deployment prevent the possibility to regularly conduct field campaigns to collect new GT data~\cite{Tardy17}.

Directly transfer a model learnt on a particular year to another period of time can be ineffective since the two time periods can be affected by different environmental, weather or climate conditions. This results in differences in the distributions of acquired remote sensing data (per year).  In the general field of machine learning, the UDA framework has the objective to provide methods and strategies to cope with data distribution shifts between the data on which the model is calibrated (\textit{source domain}) and the data on which the model is deployed (\textit{target domain})~\cite{Kouw21}. For instance, in the previous example, the goal is to have a model capable to deal with temporal transfer learning where the year on which the machine learning system is calibrated, where both remote sensing and GT data are available, is referred as \textit{source domain} while the successive year on which the machine learning model should be deployed, where only remote sensing data is available, is referred as \textit{target domain}.
While a significant amount of research work exists on UDA for the general field of computer vision and pattern recognition~\cite{Kouw21}, only few research studies were devoted to leverage such a learning setting in the context of time series of remote sensing data~\cite{Pelletier19}.

In this paper, with the aim to cope with the scenario of temporal transfer learning for SITS-based land cover mapping,  we have revised, adapted and evaluated recent state-of-the-art UDA methods. To the best of authors knowledge, this is the first work that compares recent UDA methods for temporal transfer learning for SITS land cover mapping. 
As a benchmark for the evaluation study, we have considered two Sentinel-2 SITS data describing the same study area over two different years. The study site is located in the south of France and characterized by a set of classes spanning from natural to semi-natural land covers.