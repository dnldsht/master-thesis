% Template for IGARSS-2020 paper; to be used with:
%          spconf.sty  - LaTeX style file, and
%          IEEEbib.bst - IEEE bibliography style file.
% --------------------------------------------------------------------------
\documentclass{article}
\usepackage{spconf,amsmath,epsfig}
%\usepackage{multirow}
\usepackage{comment}
\usepackage{url}
\usepackage[table,xcdraw]{xcolor}
\usepackage[normalem]{ulem}
\usepackage[mathcal]{euscript}
\newcommand{\disc}[1]{{\color{blue}$\Rightarrow$\textbf{ \textit{#1}}$\Leftarrow$}}


\useunder{\uline}{\ul}{}


% Example definitions.
% --------------------
\def\x{{\mathbf x}}
\def\L{{\cal L}}

% Title.
% ------
\title{UNSUPERVISED DOMAIN ADAPTATION METHODS FOR LAND COVER MAPPING WITH OPTICAL SATELLITE IMAGE TIME SERIES}
%
% Single address.
% ---------------
%\name{Author(s) Name(s)}
%\address{Author Affiliation(s)}
%
% For example:
% ------------
%\address{School\\
%	Department\\
%	Address}
%
% Two addresses (uncomment and modify for two-address case).
% ----------------------------------------------------------
%\twoauthors{E. Capliez, D. Ienco, N. Baghdadi}{School A-B\\
%	Department A-B\\
%	Address A-B}{R. Gaetano}{School C-D\\
%	Department C-D\\
%	Address C-D}{A. Hadj Salah}{School C-D\\
%	Department C-D\\
%	Address C-D}

\name{E. Capliez$^{1,3}$ \qquad D. Ienco$^1$ \qquad R. Gaetano$^2$ \qquad N. Baghdadi$^1$ \qquad A. Hadj Salah$^3$}

\address{${}^1$ INRAE, UMR TETIS, Univ. Montpellier, France  \\
${}^2$ CIRAD, UMR TETIS, Univ. Montpellier, France  \\
${}^3$ Airbus Defence and Space, Toulouse, France 
}

\begin{document}
%\ninept
%
\maketitle
%
\begin{abstract}
Nowadays, Satellite Image Time Series (SITS) are employed as input to derive land cover maps (LCM) to support decision makers in several application domains like agriculture and biodiversity. The generation of LCM largely relies on available ground truth (GT) data to calibrate supervised machine learning models. Unfortunately, this data are not always accessible. In this scenario, the possibility to transfer a model learnt on a particular year (\textit{source domain}) to another period of time (\textit{target domain}) could be a valuable tool to deal with the previously mentioned restrictions.

In this paper, we provide an experimental evaluation of recent Unsupervised Domain Adaptation (UDA) methods in the specific context of temporal transfer learning for SITS-based LCM. The objective is to learn a classification model at a certain year (exploiting available GT data) and, successively, transfer such a model on a subsequent year where no labelled samples are accessible. The obtained findings reveal that UDA methods represent a promising research direction to cope with the problem of temporal transfer learning for LCM. While a model learnt on the source data and directly applied on target data achieves an weighted F1-score of 67.1, the best UDA method obtains an F1-score of 83.7 with more than 15 points of positive gap. Nevertheless, there is still room for improvement that should be explored in future works.
\end{abstract}
%
\begin{keywords}
Domain Adaptation, Satellite Image Time Series, Land Cover Map, Deep Learning.
\end{keywords}
%
\section{Introduction}
\label{sec:intro}
\chapter{Introduction}

% %- timeseries classification interesting topic
% % - abundant satellite images
% Time series classification is a challenging and interesting topic that has gained significant attention over the past few years.
% With the availability of abundant satellite images, time series classification has become increasingly important in several applications such as remote sensing, climate monitoring, and environmental studies.
% The classification of time series data is a crucial task that can help extract meaningful information and insights from the data.
% % TODO expand?

Recent advances in satellite imaging technology, such as the European Union's Sentinel program, have made it possible to acquire images with a high revisit rate.
As a result, Satellite Image Time Series (SITS) data have become increasingly abundant and can be used to describe how natural and semi-natural areas evolve over time.
For example, SITS data can be used in remote sensing applications for land cover classification and change detection, climate monitoring, and environmental studies.

% TODO rephrase
However, applying supervised learning techniques to SITS data requires a reference dataset.
This means that humans must manually label the data, which can be time-consuming and labor-intensive, but is necessary to extract valuable insights and information from the data.


%- different models/methods for classification
This thesis focuses on the development and evaluation of deep learning models for the classification of time series satellite images.
The proposed models aim to classify the images accurately and efficiently, while addressing some of the challenges associated with classifying time series data.

% talk about challenges
One of the challenges we faced during our experiments was dealing with missing values in the dataset. 
Since our dataset consists of time series data of satellite images, it was expected that there would be some missing values due to weather conditions or technical issues.
Missing values can affect the performance of deep learning models because they rely on a complete set of data to learn and make accurate predictions.
Since missing values can affect the performance of our models, we approached this challenge in two ways: first, by using imputation techniques to fill in the missing values to ensure that our models could still perform well with incomplete data; second, by conducting experiments without imputation to evaluate the impact of missing values on the performance of our models.

% TODO list models?
The models used in our study include Random Forests, Temporal Convolutional Neural Networks (TempCNNs), Recurrent Neural Networks (RNNs), Generative Adversarial Networks (GANs), and Transformers.

% Talk about chapters contents:
The thesis is organized as follows: Chapter 2 provides background information on time series and deep learning models.
Chapter 3 describes the dataset used in the experiments, including its characteristics and properties.
In Chapter 4, we present our proposed method to compare the performance of different deep learning models for time series classification tasks. 
In Chapter 5, we provide a detailed description of the models used in our study, including their architectures, hyperparameters, training procedures, experiments, and the results we obtained from these experiments.
Finally, in Chapter 6, we present our conclusions and suggest possible future work in the area of time series classification.

\section{Related work}
\label{sec:relatedwork}
The following section presents an overview of the state of the art methods in the field of unsupervised domain adaptation (UDA). In the UDA setting, we dispose of data from a source domain $\mathcal{D}_s$, with associated label information and unlabelled data from a target domain $\mathcal{D}_t$. The main assumptions are the follows: i) there is a shift in the data distribution between $\mathcal{D}_s$ and $\mathcal{D}_t$ and ii) $\mathcal{D}_s$ and $\mathcal{D}_t$ share the same label space (homogeneous domain adaptation). The goal is to build a classification model capable to exploit, simultaneously, the information associated to the source domain (data and labels) and the available information associated to the target domain (data) to perform prediction on the unlabelled data belonging to $\mathcal{D}_t$. For a general overview of unsupervised domain adaptation approaches please refer to~\cite{Kouw21}.

\subsection{State-of-the-art unsupervised DA methods}
\label{ssec:sotauda}

\textit{Geodesic Flow Kernel (GFK)~\cite{6247911}:} This approach aligns the source and target data distributions by means of a geodesic flow kernel-based strategy. The method allows to project both source and target data into a shared, low-dimensional, space in which the distribution shift between the two domains is reduced. Successively, any standard supervised classification method can be trained on the source data and tested on the target ones.

\textit{Domain-Adversarial Neural Networks (DANN) \cite{ganin2016domain}:} This approach enrich a standard neural network-based supervised classification strategy with a domain classifier that may distinguish between source and target examples.
The domain classifier is associated with a gradient reverse layer (GRL) that enforces the features extracted by the encoder to be invariant w.r.t. the distribution shift that can be present between $\mathcal{D}_s$ and $\mathcal{D}_t$.

\textit{Adversarial Discriminative Domain Adaptation (ADDA) \cite{8099799}:} Inspired by the concept of generative adversaial network (GAN), this approach set up a two players learning strategy where a discriminator network tries to distinguish between source and target data representation provided by the generator while the generator tries to fool the discriminator network. Also in this case, the objective is to extract data representations that are invariant w.r.t. possible distribution shifts that can occur between data coming from $\mathcal{D}_s$ and $\mathcal{D}_t$.


\subsection{DA of Satellite Image Time Series}
\label{ssec:dasits}
Although there is a significant amount of work on UDA, only a limited number of studies are devoted to cope with time series data~\cite{WilsonDC20} and, even less for satellite image time series~\cite{9324339}. For the latter, the work proposed in~\cite{9324339} clearly underlines that state-of-the-art UDA methods cannot directly deal with spatial transfer learning while no research study, to the best of our literature survey, has evaluated the quality of recent UDA methods in the context of temporal transfer learning for land cover mapping based on SITS data.






\section{Data}
\label{sec:data}
The study area is centered on the city of Balaruc-les-Bains in south-eastern France. It borders the Mediterranean Sea and it has a surface of 100 $ km^2 $. It is a small but dense urban area surrounded by agricultural crops - mostly vineyards - and natural vegetation. Figure~\ref{fig:gtmap} presents the GT map superposed on the Sentinel-2 image of the 23th January 2018. 

\begin{figure}[htb]
\begin{minipage}[b]{1.0\linewidth}
  \centering
    \centerline{\epsfig{figure="Figures/BLC_20180123_SAT-GT_300dpi.png",width=8cm}}
\end{minipage}
\caption{Ground truth data superposed to a Sentinel-2 image.}
\label{fig:gtmap}
\end{figure}


\textit{Satellite Image Time Series:} We collect satellite image time series of Sentinel-2 imagery spanning the years 2018 and 2019. For each year, we retain 24 images to guarantee a similar temporal sampling step. Images were also chosen with the aim to be representative of the temporal (annual) evolution of the land covers associated to the study area as well as to filter out images that were seriously impacted by cloud phenomena. Figure~\ref{fig:chronology} depicts the acquisition dates of the two Sentinel-2 satellite image time series. 

\begin{figure}[htb]
\begin{minipage}[b]{1.0\linewidth}
  \centering
    \centerline{\epsfig{figure="Figures/2018-2019_chronologie_EN.pdf",width=8.5cm}}
\end{minipage}
\caption{Acquisition dates of each Satellite Image Time Series.}
\label{fig:chronology}
\end{figure}

All images were provided by the THEIA pole platform~\footnote{\url{http://theia.cnes.fr}} at level-2A in top of canopy reflectance values with associated cloud masks. Only 10-m spatial resolution bands (Blue, Green, Red and Near infrared spectrum) were considered in this analysis. A preprocessing was performed over each band to replace cloudy pixel values as detected by the available cloud masks through a linear multi-temporal interpolation (cf. temporal gap-filling~\cite{IENCO201911}). The entire study site is enclosed in the Sentinel-2 tile 31TEJ with relative orbit n°8.

\textit{Ground truth data:} The GT data for 2018 was built from various sources: the \textit{Registre Parcellaire Graphique} (RPG) reference data (the French land parcel identification system), the French National Geographic Institute \textit{‘BD-Topo \& BD Forêt’} and the visual interpretation of a SPOT 6 image (to assess and enrich the GT data) as well. The GT was assembled in Geographic Information System (GIS) vector file, containing a collection of polygons, each attributed with a land cover category. Statistics about the GT data are reported in Table~\ref{tab:gt}.
It was assumed that this GT data is also valid for 2019 because no significant change has occurred in this area between 2018 and 2019. 

\begin{table}[htb]
\centering
\scriptsize
\begin{tabular}{ |l|c|r|r| }
 \hline
 Class Name & Class ID. & \# Polygons & \# Pixels \\ 
 \hline
 BUILT & 1 & 712 & 10411\\
 ROAD & 2 & 328 & 5165 \\
 WATER & 3 & 82 & 32095 \\  
 FOREST & 4 & 172 & 49175 \\  
 VINE & 5 & 223 & 28630 \\
 ORCHARD & 6 & 32 & 2075 \\
 CROP & 7 & 39 & 5205 \\
 OTHER VEGETATION & 8 & 56 & 4812 \\
 \hline
 & Total & 1644 & 137568 \\
 \hline
\end{tabular}
\caption{Ground truth statistics.}
\label{tab:gt}
\end{table}

\section{Experiments}
\label{sec:experiments}
\section{Experiments}

An introduction here (with a cite from bibliography, like this: \cite{greenwade93}).


\section{Conclusion}
\label{sec:conclusion}
\chapter{Conclusions}

Conclusions \& Future work

- Graph neural networks\\
- Im-BiLSTM\\
 
% References should be produced using the bibtex program from suitable
% BiBTeX files (here: strings, refs, manuals). The IEEEbib.bst bibliography
% style file from IEEE produces unsorted bibliography list.
% -------------------------------------------------------------------------
\bibliographystyle{IEEEbib}
%\small
\bibliography{refs}

\end{document}
